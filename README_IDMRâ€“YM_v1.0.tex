\documentclass[12pt]{article}
\usepackage[utf8]{inputenc}
\usepackage{amsmath, amssymb}
\usepackage{geometry}
\usepackage{hyperref}
\usepackage{graphicx}
\usepackage{enumitem}
\geometry{margin=1in}

\title{IDMR--YM: Induced Mass and Metric Rescaling Framework}
\author{
Victor Eduardo Morales Córdoba \\
Email: \texttt{vmorales@uned.cr} \\
ORCID: \href{https://orcid.org/0009-0000-8787-6141}{0009-0000-8787-6141}
}

\date{Versión 1.0 -- Octubre 2025}

\begin{document}

\maketitle

\section*{�� Resumen Ejecutivo}
\textbf{IDMR--YM} es un marco teórico unificado que reinterpreta el mecanismo de masa fermiónica y la geometría del espacio-tiempo mediante \textbf{saturación escalar} y \textbf{métrica dinámica}. Combina conceptos de teoría gauge, relatividad general y campos escalares para ofrecer una descripción geométrica de la masa inducida.

\section*{�� Fundamento Teórico}
\begin{itemize}
  \item \textbf{Masa Inducida}: Los fermiones adquieren masa mediante acoplamiento a un campo escalar de saturación \( \phi \)
  \item \textbf{Métrica Dinámica}: El campo escalar modula la geometría del espacio-tiempo vía \( g_{\mu\nu}(\phi) = \eta_{\mu\nu} f(\phi) \)
  \item \textbf{Conexión Extendida}: Unificación de transporte gauge y modulación escalar
\end{itemize}

\textbf{Lagrangiano Fundamental:}


\[
\mathcal{L}_{\text{IDMR--YM}} = \bar{\psi}(i\gamma^\mu D_\mu - m(\phi))\psi + \frac{1}{2}g^{\mu\nu}(\phi)\partial_\mu \phi \partial_\nu \phi - V(\phi) - \frac{1}{4}F^{\mu\nu}F_{\mu\nu}
\]



\section*{�� Ecuaciones de Campo Derivadas}
\subsection*{1. Ecuación de Dirac Modificada}


\[
(i\gamma^\mu D_\mu - m(\phi) + \lambda\phi^2)\psi = 0
\]


\textbf{Donde:}
\begin{itemize}
  \item \( D_\mu = \partial_\mu + ieA_\mu + \Gamma_\mu(\phi) \)
  \item \( m(\phi) = \alpha\phi^2 + \beta\partial_\mu\phi\partial^\mu\phi \)
  \item \( \Gamma_\mu(\phi) \) modula fase y dispersión fermiónica
\end{itemize}

\subsection*{2. Ecuación Escalar de Saturación}


\[
\Box\phi + \frac{\partial V}{\partial\phi} - 2\alpha\phi\bar{\psi}\psi - \beta\partial_\mu(\bar{\psi}\psi\partial^\mu\phi) + \frac{1}{2}\frac{\partial f}{\partial\phi}T^{\mu\nu}_{(\psi)}\eta_{\mu\nu} = 0
\]



\subsection*{3. Ecuaciones de Einstein Modificadas}


\[
G_{\mu\nu}(g(\phi)) = 8\pi G\left(T^{(\psi)}_{\mu\nu} + T^{(\phi)}_{\mu\nu} + T^{(\text{gauge})}_{\mu\nu}\right)
\]


\textbf{Con métrica dinámica:} \( g_{\mu\nu}(\phi) = \eta_{\mu\nu}f(\phi) \), donde \( f(\phi) = 1 + \epsilon\phi^2 + O(\phi^4) \)

\section*{�� Predicciones Testables}
\begin{center}
\begin{tabular}{|l|l|l|}
\hline
\textbf{Observable} & \textbf{Predicción IDMR--YM} & \textbf{Método de Verificación} \\
\hline
Masa del electrón & \( m_e = \alpha\langle\phi\rangle^2 + \beta(\partial\phi)^2 \) & Ajuste a \( m_e = 511 \text{keV} \) \\
Factor g--2 & \( \Delta a_e \approx \frac{\epsilon}{8\pi^2}\frac{m_e^2}{m_\phi^2} \) & Comparación con \( a_e^{\text{exp}} \) \\
Expansión cósmica & \( H(z) \sim H_0\sqrt{f(\phi(z))} \) & Datos de supernovas Type Ia \\
Ondas gravitacionales & Modos escalares adicionales \( h_{+,\times} + \phi \) & LISA, Einstein Telescope \\
Constantes variantes & \( \alpha_{\text{EM}}(\phi) \propto f^{-1/2}(\phi) \) & Relojes atómicos, espectros lejanos \\
\hline
\end{tabular}
\end{center}

\section*{�� Ejemplos Numéricos}
\subsection*{Masa Inducida Electrónica}
Para \( \langle\phi\rangle = 246 \text{GeV} \) y \( \alpha \approx 10^{-11} \):


\[
m_e = \alpha\langle\phi\rangle^2 \approx 0.511 \text{MeV}
\]



\subsection*{Corrección a g--2}
Con \( \epsilon = 0.1 \), \( m_\phi = 1 \text{TeV} \):


\[
\Delta a_e \approx 2.5\times 10^{-14}
\]



\subsection*{Modulación Métrica}


\[
f(\phi) = 1 + 0.01\phi^2 \Rightarrow H(z) \approx 1.005H_0\sqrt{\Omega_m(1+z)^3 + \Omega_\Lambda}
\]



\section*{�� Conexión con Módulos IDMR}
\begin{itemize}
  \item \textbf{GRB--IDMR v2.1}: Expansión métrica en campos gravitacionales fuertes
  \item \textbf{GPS--IDMR v2.3}: Sincronización relativista con modulación escalar
  \item \textbf{Saturación Escalar IDMR}: Base conceptual para \( m(\phi) \) y acoplamientos
\end{itemize}

\section*{�� Citas y Referencias}
\begin{verbatim}
@software{IDMR-YM,
  author = {Victor Eduardo Morales Córdoba},
  title = {IDMR--YM: Induced Mass and Metric Rescaling Framework},
  doi = {10.5281/zenodo.XXXXXXX},
  version = {1.0},
  publisher = {Zenodo},
  year = {2025}
}
\end{verbatim}

\section*{�� Estado del Proyecto}
\begin{itemize}
  \item ✅ Completado: Formulación teórica, ecuaciones de campo
  \item �� En Desarrollo: Implementación numérica, validación fenomenológica
  \item �� Próximamente: Extensiones no-abelianas, cosmología completa
\end{itemize}

\end{document}
